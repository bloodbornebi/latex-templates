\documentclass[lowerhead,12pt]{aesthetic}
\usepackage[utf8]{inputenc}
\usepackage{enumitem}
\usepackage{afterpage}
\usepackage{hyperref}
\usepackage{varioref}
\usepackage[nameinlink]{cleveref}
\usepackage[xindy]{imakeidx}
\usepackage{xcolor}
\usepackage{pdflscape}
\usepackage[chapter]{minted}
\usemintedstyle{solarized-light}

\makeindex

\title{packages and classes container}
\author{matthew andrews}
\date{november 2020}

\setsecnumdepth{subparagraph}
\maxsecnumdepth{subparagraph}

\begin{document}
\frontmatter
\maketitle
\thispagestyle{empty}
\clearpage

\contents
\listoflistings

\mainmatter
\part{classes}
\chapter{aesthetic}
\section{quick facts}
\begin{description}
    \item[file] \texttt{aesthetic.cls}
    \item[description] a memoir based class for providing the quick style i prefer
    \item[dependencies] none, can be fragile regarding styling/fonts/etc
    \item[requires] \begin{description}[font=\normalfont\scshape\ttfamily]
        \item[Alegreya] optional --- see \vref{sec:aesop}
        \item[atbegshi] allows for appendix command --- see \vref{sec:macro}
        \item[array] needed for the \texttt{M} column type\footnote{i believe this is called by \texttt{memoir} anyway, but i have no interesting breaking things/any dependencies}
        \item[microtype] no current way to pass options, and i don't intend on adding one. the point is to make documents look nice as painlessly as possible
        \item[pifont] needed for fancy breaks
    \end{description}
    \item[loads] {\scshape\ttfamily memoir} --- options cannot be overridden and are \begin{description}[font=\ttfamily]
        \item[openany] only has a super noticable effect if \texttt{pagebreaks} or \texttt{partpages} is called (\vref{sec:aesop})
        \item[twoside] this is designed for book like classes, and the headings will break if you call oneside
        \item[fullptlayout] just helps typesetting, especially since we call {\scshape\ttfamily microtype}\footnote{i'm honestly not 100\% sure this makes a difference, but i'm not turning it off}
        \item[extrafontsizes] so you can call all of the large text sizes you want
        \item[[user options{]}] whatever you want. \texttt{twocolumn} is disabled (will break the page layout, and two column documents are plenty compact. this is to help with the tightness of one column documents, because presumably you want decent readability without using too much paper.)
    \end{description}
    \item[example] this file (using \texttt{lowerhead}, \texttt{12pt})
    \item[last updated] november 23 , 2020
\end{description}

\section{class options}\label{sec:aesop}
\begin{description}[font=\ttfamily]
    \item[lowerhead] formats head in small caps/lowercase, vs all caps. not recommended for part headings.\footnote{there is currently no way to change this within the document. i have no plans of changing this.}
    \item[numberedhead] adds numbers to headings\footnotemark[\thefootnote]
    \item[otherfont] allows a font other than Alegreya to be called. if you call this, Alegreya just isn't loaded, so if you want to use Alegreya for headers/what have you, call the font and change it manually.
    \item[pagebreaks] adds pagebreaks/supresses the commands which condense the document. this does \emph{not} reformat part parges
    \item[partheadings] changes headings from being labeled as chapter \& section to part \& chapter\footnotemark[\thefootnote]
    \item[partpages] allows parts to be formatted on their own page, while maintaining the fancy breaks. if enabled with \texttt{pagebreaks}, will provide the package partpage formatting
    \item[shortnames] turns off the automatic full part/chapter names in the header. unless \texttt{numberedhead} is called, this has no effect.
\end{description}

\section{related/recommended}
\begin{description}[font=\normalfont\scshape\ttfamily]
    \item[afterpage] helps with float and table typesetting
    \item[booktabs] the aesthetics of this make heavy use of horizontal rules, so you'll probably want to have the nicer looking tables anyway
    \item[csquotes] i think either {\scshape\ttfamily babel} or {\scshape\ttfamily memoir} calls this, and it'll make your life a bit easier
    \item[xcolor] colors are nice, they jive well, and if you want to mess with some of the {\scshape\ttfamily memoir} opptions, now you can.
    \item[xtab] helps with longer tables. you can use {\scshape\ttfamily longtable} if that's your preference, but if i ever write up some of my other common commands as classes/packages, i'll write them for {\scshape\ttfamily xtab}
\end{description}

\section{macros}\label{sec:macro}
\begin{description}[font=\ttfamily]
    \item[\textbackslash contents] calls the standard set of contents (short contents, full contents, list of figures, list of tables). adds a part for contents, but does not enter or exist \verb|\frontmatter| so you can add contents under. not very flexible.
    \item[M\{<width>\}] a horizontally/vertically centered column.
    \item[\textbackslash makeappendix] prints out a formmatted appendix. see the end of this document.
    \item[\textbackslash makebackmatter{[]}] prints out the command to enter backmatter without forcing a page break. optional argument typesets a file/text between the switch command and the transition spacer.
\end{description}

\section{known issues}
\begin{description}
    \item[{\scshape\ttfamily afterpage} breaking] yeah this isn't really a class problem \emph{however} i bring it up here because it does happen, where you get fucked over because {\scshape\ttfamily afterpage} doesn't know what it's doing. just make sure you've added lots of text.
    \item[calling footnotes in sections] something about the formmatting commands for the sections breaks footnotes\footnote{because of the commands that format them, sections are actually pretty fragile. there isn't a consistent and easy workaround.}. you can't use \verb|\protect\footnote{}| because it will incorrectly format the font of the footnote. instead, use \verb|\protect\footnotemark}\footnotetext{}|. this does not happen for chapters.
    \item[failure to break pages] you may have used \verb|\let\par\russianpar|. bad idea. it will mess with the commands used to control chapters. this doesn't always cause problems (especially if you have longer chapters), but i don't recommend messing with it.
    \item[weird looking tables] explained in \vref{sec:aesinfo}. basically, you should not call flexible width tables in this, and instead, use \texttt{M}
\end{description}

\section{possible additions/plans}
\begin{description}
    \item[footnote toggles] turning where footnotes are numbered into an option, as well as adding easier per page symbolic footnotes.
    \item[upper/lower options] so that for the $\sim$aesthetic$\sim$ all lowercase documents can also have easy tables of contents, etc.
\end{description}

\section[class information]{class information\protect\footnotemark}\footnotetext{not recommended, but i am a completionist in all things}\label{sec:aesinfo}
\begin{description}
    \item[class options] pretty much all of the class options just create conditionals in the form \verb|\@option<boolean>| which can be used with \verb|\makeatletter| and \verb|\makeatother|\footnote{i'm not sure if this would actually work, however, because none of the class options have anything to do with the document once you've started the document, they're all called on start up. i guess this could change in the future?}.
    \item[headings] we use a modified version of the \texttt{ruled} style which gives us some flexibility in the marks. this bit of the file is more verbose than needed, but if you want to mess with the headings, you can mess with the \texttt{ruled} style
    \item[parts and chapters] basically just heavy use of \verb|\renewcommand{}{}| over breaks, skips, fonts, etc., to get it all consistent.
    \item[tabcolsep] set to \texttt{0pt}, so make sure you consider that. this is, as far as i'm concerned, a positive for typesetting tables because it lets you actually use textwidth based ratios for setting tables.
    \item[footnotes] numbered consecutively though the whole document. could easily be changed in the preamble, but i don't recommend doing that unless you're using \texttt{pagebreaks}, because it can create weirdness like two footnotes of the same number on the same page in edge cases.
    \item[contents] just calls the contents in the predictable/defined way.
    \item[appendix] calls a few commands to add everything to the table of contents without being a pain
    \item[backmatter] uses {\scshape\ttfamily atbegshi} to turn \texttt{\textbackslash @mainmatterfalse} on after the page finishes\footnote{for grouping reasons, \texttt{\textbackslash afterpage} doesn't wrok here, and the work around basically becomes comes just using what {\scshape\ttfamily atbegshi} does.}, and then after the optional argument, adds a page separator
\end{description}

\chapter{notes}
\section{quick facts}
\begin{description}
  \item[file] \texttt{notes.cls}
  \item[description] an extension of {\scshape\ttfamily aesthetic.cls} which provides a variety of tools for notes
  \item[dependencies] \begin{itemize}
      \item {\scshape\ttfamily aesthetic} class is required
      \item if using class option \texttt{cse} (\vref{sec:notesop}), {\scshape\ttfamily minted} must be configured. in order for this to work correctly, the color schemes for \texttt{solarized}, \texttt{nord}, etc., must be loaded in under the correct names. this is fully described under \nameref{sec:notesop} (\vref{sec:notesop}).
    \end{itemize}
  \item[requires]  \begin{description}
      \item[general] \begin{description}[font=\normalfont\scshape\ttfamily]
          \item[xcolor] required to manage the themes, etc.
          \item[tcolorbox] used in effectively every course type to provide functionality
          \item[float] provides a variety of useful features
          \item[caption] allows for more flexible caption positioning/placement, necessary for various functions
          \item[tikz] commonly used in notes, included here for ease. also, it's required in a variety of places, so requiring it early prevents dependency issues.
          \item[hyperref, varioref, cleveref] provides effective refrencing and linking. hyperref also allows link color to be set to match the document color scheme
        \end{description}
      \item[anth] \begin{description}[font=\normalfont\scshape\ttfamily]
          \item[biocon] provides support for species names
        \end{description}
      \item[chem] \begin{description}[font=\normalfont\scshape\ttfamily]
          \item[chemfig] draws chemical figures
          \item[mhchem] (version 4) better inline chemistry figures
          \item[amsmath, amssymb, amsthm] provides common math support, etc.
          \item[thmtools] provides tools for reaction mechanism environment
        \end{description}
      \item[cse] \begin{description}[font=\normalfont\scshape\ttfamily]
          \item[minted] provides listing support. color themes are automatically loaded to match document theme.\footnote{if using \texttt{printed} class option (\vref{sec:notesop}), \texttt{bw} is automatically loaded. color schemes without minted support (presently, \texttt{ashes}) have themes that are mostly matching chosen.}
        \end{description}
      \item[math] \begin{description}[font=\normalfont\scshape\ttfamily]
          \item[amsmath, amssymb, amsthm] provide common/basic math support
          \item[thmtools] provides better theorem environments (necessary to provide boxed theorems)
          \item[physics] provides helpful math commands
          \item[jkmath] provides similar functionality to {\scshape\ttfamily physics}\footnote{both are loaded because of document compatibility and because physics provides a few things that jkmath doesn't.}
        \end{description}
      \item[ling] \begin{description}[font=\normalfont\scshape\ttfamily]
          \item[tipa] basic ipa support
          \item[tipx] extended ipa support
          \item[gb4e] glossing
          \item[forest] a package for creating tikz trees easier/more effectively\footnote{the tikz library trees is also loaded, but forest provides better linguistics functionality and is generally much easier to use}
        \end{description}
    \end{description}
    \item[example] not presently loaded
    \item[last updated] december 27, 2020
\end{description}

\section{class options}\label{sec:notesop}
\begin{description}[font=\ttfamily]
  \item[dark] loads a dark version of a theme (or, in a black and white color scheme, simply provides white on black)
  \item[printable] \begin{table}
      \centerfloat
      \begin{tabular}{M{.25\textwidth}M{.4\textwidth}M{.35\textwidth}}
        \toprule
        component & standard & change \\
        \midrule
        font size & 14pt & 10pt \\
        dark & changes document to dark & negated \\
        page color & light/dark color & white \\
        background color & darker/lighter color & white \\
        {\scshape\ttfamily minted} scheme & matches color scheme & \texttt{bw} \\
        links & colored (prints) & colored boxes (doesn't print) \\
        \bottomrule
      \end{tabular}
      \caption{changes made in printable}
      \label{tab:printable}
    \end{table}
    implements the changes listed in \vref{tab:printable}
  \item[ashes, espresso, nord, solarized, tomorrow] \begin{landscape}\begin{table}
      \centerfloat
      \begin{tabular}{M{.1\linewidth}M{.1\linewidth}M{.05\linewidth}M{.1\linewidth}M{.1\linewidth}M{.1\linewidth}M{.1\linewidth}M{.1\linewidth}M{.1\linewidth}M{.1\linewidth}M{.05\linewidth}}
        \toprule
        scheme & text & page & back\-ground & subtle text & accent & accent text& se\-con\-dary ac\-c\-ent & se\-con\-dary ac\-c\-ent text & ter\-ti\-a\-ry ac\-c\-ent & fourth \\
        \midrule
        ash\-es (light) & {\color{white} \colorbox[HTML]{1c1e23}{dark}} & \colorbox[HTML]{f3f3f5}{light} & \colorbox[HTML]{f3f3f5}{light} & {\color{white} \colorbox[HTML]{747a84}{mid}} & {\color{white} \colorbox[HTML]{ac95c7}{a1}} & \colorbox[HTML]{f3f3f5}{light} & {\color{white} \colorbox[HTML]{95acc7}{b2}} & {\color{white} \colorbox[HTML]{1c1e23}{dark}} & \colorbox[HTML]{c7ac95}{a0} & \colorbox[HTML]{acc795}{b1} \\
        ashes (dark) & \colorbox[HTML]{f3f3f5}{light} & {\color{white} \colorbox[HTML]{1c1e23}{dark}} & {\color{white} \colorbox[HTML]{1c1e23}{dark}} & {\color{white} \colorbox[HTML]{747a84}{mid}} & {\color{white} \colorbox[HTML]{ac95c7}{a1}} & \colorbox[HTML]{f3f3f5}{light} & {\color{white} \colorbox[HTML]{95acc7}{b2}} & {\color{white} \colorbox[HTML]{1c1e23}{dark}} & \colorbox[HTML]{c7ac95}{a0} & \colorbox[HTML]{acc795}{b1} \\
        es\-pre\-sso (light) & {\color{white} \colorbox[HTML]{323232}{dark0}} & \colorbox{white}{light1} & \colorbox[HTML]{eeeeec}{light0} & {\color{white} \colorbox[HTML]{353535}{dark1}} & {\color{white} \colorbox[HTML]{d25252}{a0}} & \colorbox{white}{light1} & \colorbox[HTML]{6c99bb}{b0} & {\color{white} \colorbox[HTML]{353535}{dark1}} & \colorbox[HTML]{c2e075}{a4} & \\
        es\-pre\-sso (dark) & \colorbox[HTML]{eeeeec}{light0} & {\color{white} \colorbox[HTML]{323232}{dark0}} & {\color{white} \colorbox[HTML]{353535}{dark1}} & \colorbox[HTML]{eeeeec}{light0} & {\color{white} \colorbox[HTML]{d25252}{a0}} & {\color{white} \colorbox[HTML]{323232}{dark0}} & \colorbox[HTML]{6c99bb}{b0} & {\color{white} \colorbox[HTML]{353535}{dark1}} & \colorbox[HTML]{c2e075}{a4} & \\
        nord (light) & {\color{white} \colorbox[HTML]{2E3440}{nord0}} & \colorbox[HTML]{ECEFF4}{nord6} & \colorbox[HTML]{E5E9F0}{nord5} & {\color{white} \colorbox[HTML]{3B4252}{nord1}} & \colorbox[HTML]{81A1C1}{nord9} & \colorbox[HTML]{eceff3}{nord6} & \colorbox[HTML]{A3BE8C}{nord14} & \colorbox[HTML]{e5e9f0}{nord5} & \colorbox[HTML]{B48EAD}{nord15} & \\
        nord (dark) & \colorbox[HTML]{ECEFF4}{nord6} & {\color{white} \colorbox[HTML]{2E3440}{nord0}} & {\color{white} \colorbox[HTML]{3B4252}{nord1}} & &  \colorbox[HTML]{81A1C1}{nord9} & \colorbox[HTML]{eceff3}{nord6} & \colorbox[HTML]{A3BE8C}{nord14} & \colorbox[HTML]{e5e9f0}{nord5} & \colorbox[HTML]{B48EAD}{nord15} & \\
        so\-l\-ar\-iz\-ed (light) & {\color{white} \colorbox[HTML]{002b36}{base03}} & \colorbox[HTML]{fdf6e3}{base3} & \colorbox[HTML]{eee8d5}{base2} & {\color{white} \colorbox[HTML]{073642}{base02}} & \colorbox[HTML]{b58900}{yellow} & \colorbox[HTML]{fdf6e2}{base3} & \colorbox[HTML]{268bd2}{blue} & \colorbox[HTML]{eee8d5}{base2} & \colorbox[HTML]{dc322f}{red} & \\
        so\-l\-ar\-iz\-ed (dark) & \colorbox[HTML]{fdf6e3}{base3} & {\color{white} \colorbox[HTML]{002b36}{base03}} & {\color{white} \colorbox[HTML]{073642}{base02}} & \colorbox[HTML]{eee8d5}{base2} & \colorbox[HTML]{b58900}{yellow} & {\color{white} \colorbox[HTML]{002b3b}{base03}} & \colorbox[HTML]{268bd2}{blue} & {\color{white} \colorbox[HTML]{073642}{base02}} & \colorbox[HTML]{dc322f}{red} & \\
        to\-mor\-row (light) & {\color{white} \colorbox[HTML]{1d1f21}{dark0}} & \colorbox[HTML]{ffffff}{light3} & & & \colorbox[HTML]{cc6666}{a0} & & & & & \\
        to\-mor\-row (dark) & \colorbox[HTML]{ffffff}{light3} & {\color{white} \colorbox[HTML]{1d1f21}{dark0}} & & & \colorbox[HTML]{cc6666}{a0} & & & & & \\
        \bottomrule
      \end{tabular}
      \caption[color schemes]{color schemes. note that most color schemes have 10 to 16 colors, but only 5 or so are used.}
      \label{tab:colorschemes}
    \end{table}
  \end{landscape}
  document color schemes. see \vref{tab:colorschemes}. note that these are mutually exclusive and will be ordered alphabetically (e.g., if you include both \texttt{solarized} and \texttt{nord}, solarized will be displayed, but if you include \texttt{espresso} and \texttt{nord}, nord will be displayed).

  \item[anth, chem, cse, ling, math] course types. differences are discussed in relevant sections, but mostly pertains to which packages are loaded and which macros are created. these are not exclusive, but certain macros may behave unpredictably.
\end{description}

\section{related/recommended}
this strives to be relatively self-contained, so only packages required for personal needs must be loaded.

\section{macros}
\subsection{general}
\begin{description}[font=\ttfamily]
  \item[flashcard\footnotemark]\footnotetext{not technically a macro} a way of maintaining blocks that may be desired for flashcards. should not be directly used.

  \item[vocab] environment. takes one mandatory argument (vocabulary word), and the contents are placed in a box.

  \item[startgen] prints a title page and contents. title information must be set in preamble. takes one optional argument, which will be placed at the end of the contents. always usable, but there are better content/title page options listed in \vref{subsec:chem,subsec:cse}.
\end{description}

\subsection{chemistry}\label{subsec:chem}
\begin{description}[font=\ttfamily]
  \item[mechanism] a float type with provides reaction mechanism support. not recommended for direct use.

  \item[mech] an environment which takes one argument (title) and boxes a reaction mechanism.

  \item[start] chemistry's implementation of start adds a list of reaction mechanisms. it also takes an optional argument.
\end{description}

\subsection{computer science}\label{subsec:cse}
\begin{description}[font=\ttfamily]
  \item[code] a listing template. takes one argument, the title of the box.

  \item[start] computer science's implementation of start includes a list of flashcards and a list of listings.
\end{description}

\subsection{math}
\begin{description}[font=\ttfamily]
  \item[theorem, axiom, definition] theorem types defined. not recommended for direct usage because they're formatted very precisely to allow for the boxed variants to function in a predictable way.

  \item[boxtap] an environment for theorems, axioms, and postulates\footnote{i just didn't like boxtad as an acronym }. can be used directly, but not recommended. takes two arguments: the first identifies which box style to use, and the second is the title. if you need custom theorems, better to use this in another theorem definition than directly.
  \begin{listing}
    \begin{minted}[breaklines]{TeX}
\tcbset{theorem/.style={boxtap, colframe=secondaryaccentcolor, colbacktitle=secondaryaccentcolor}}
\declaretheorem[style=boxedtheorem]{theorem}
\NewDocumentEnvironment{ boxthm }{ m m +b }{ \begin{boxtap}{theorem}{\Cref{#2} (\nameref{#2})}\begin{theorem}[name={#1},label=#2] #3 \end{theorem}\end{boxtap} }{}
    \end{minted}
    \caption[example of \texttt{boxtap} implementation]{example implementation of \texttt{boxtap}. note that \texttt{boxtap} and \texttt{boxedtheorem} are created within the class, you can use these freely.}
    \label{lst:boxtap}
  \end{listing}
  see \vref{lst:boxtap} for an example of how to do this.

  \item[boxthm, boxaxm, boxdef] default variants of \texttt{boxtap}. all have unique colors. see \vref{lst:boxtap} for most of \texttt{boxthm}'s application.
\end{description}

\subsection{linguistics}
\begin{description}[font=\ttfamily]
  \item[start] provides a contents line with a list of flashcards. does not include examples.
  \item[tree] a document environment for syntax trees. takes one argument, the caption, and places body directly into a tree within a floating environment.
\end{description}

\section{known issues}
\begin{description}
  \item[bad flashcard handling] for reasons that are somewhat difficult to explain concisely, flashcards can behave very unpredictably.
  \item[\texttt{color scheme} is broken] as \vref{tab:colorschemes} shows, some color schemes lack certain colors. this is for a variety of reasons, most of them bad, but it is relevant to mention that in most cases, there should be a back up of either white or black.
\end{description}

\section{possible additions/plans}
\begin{description}
  \item[return buttons on links] obviously this will be rather imperfect, but being able to return from a vocab card to where it was mentioned in the main document would be nice.
  \item[improved example functionality] at present, the \mintinline{TeX}{example} environment for linguistics is minimal at best. ideally, this could be improved to function as a full float.
  \item[better internal handling of flashcards] flashcards are a mess of spaghetti code right now which really limits their power. cleaning this up would make most things dealing with flashcards significantly more effective.
  \item[key-value colors] exactly what it says on the tin.
  \item[more flexible minted processing] avoiding errors from missing minted options, possibly (?) loading a back-up package if minted can't be run
  \item[flashcard export] ideally in a CSV file\footnote{ideal for anki support}
\end{description}

\section{class information}
\begin{description}
  \item[xparse] loaded in very early on. could be loaded in earlier, but this provides support for future key-value work.
  \item[class options] very similar to aesthetic, most are loaded in to create if statements in the general form \mintinline{tex}{\if@dark}, etc. the options \mintinline{tex}{partheadings}\footnote{for notes headings to be helpful, they generally need to be more specific} and \mintinline{TeX}{lowerhead}\footnote{automatically loaded} are explicitly ignored.
  \item[minted] this is only loaded in the cse version, but the thematic handling is a series of nested if statements --- \vref{lst:minted}
  \begin{listing}
    \begin{minted}[breaklines]{TeX}
\usemintedstyle{bw}
\if@printable
  \usemintedstyle{bw}
\else
  \if@ashes
    \if@dark
      \usemintedstyle{native}
    \else
      \usemintedstyle{colorful}
    \fi
  \fi

  \if@espresso
    \usemintedstyle{nord}
  \fi

  \if@nord
    \usemintedstyle{nord}
  \fi

  \if@solarized
    \if@dark
      \usemintedstyle{solarized-dark}
    \else
      \usemintedstyle{solarized-light}
    \fi
  \fi

  \if@tomorrow
    \if@dark
      \usemintedstyle{tomorrow-night}
    \else
      \usemintedstyle{tomorrow}
    \fi
  \fi
\fi
    \end{minted}
    \caption{loading minted styles based on user choice}
    \label{lst:minted}
  \end{listing}
  \item[colors] the code for the ashes scheme is shown in \vref{lst:ashes} to provide an example of what the color scheme setting is. the other colors mostly follow the same pattern.
  \begin{listing}
    \begin{minted}{tex}
\definecolor{@dark}{HTML}{1c1e23}
\definecolor{@light}{HTML}{f3f3f5}
\definecolor{@mid}{HTML}{747a84}
\definecolor{@a0}{HTML}{c7ac95}
\definecolor{@a1}{HTML}{ac95c7}
\definecolor{@a2}{HTML}{c795ac}
\definecolor{@b0}{HTML}{95c7ac}
\definecolor{@b1}{HTML}{acc795}
\definecolor{@b2}{HTML}{95acc7}

\definecolor{subtletextcolor}{named}{@mid}
\definecolor{accentcolor}{named}{@a2}
\definecolor{accenttextcolor}{named}{@light}
\definecolor{secondaryaccentcolor}{named}{@b2}
\definecolor{secondaryaccenttextcolor}{named}{@dark}
\definecolor{tertiaryaccentcolor}{named}{@a0}
\definecolor{color4}{named}{@b1}

\if@dark
  \if@printable
    \definecolor{textcolor}{named}{@dark}
  \else
    \definecolor{textcolor}{named}{@light}
    \definecolor{pagecolor}{named}{@dark}
    \definecolor{backgroundcolor}{named}{@dark}
  \fi
\else
  \definecolor{textcolor}{named}{@dark}
  \definecolor{backgroundcolor}{named}{@light}

  \if@printable
  \else
    \definecolor{pagecolor}{named}{@light}
  \fi
\fi
    \end{minted}
    \caption{ashes color scheme}
    \label{lst:ashes}
  \end{listing}

  \item[other color choices] other color decisions are made based on the color names defined here (accent colors, etc.) which are available for use in the main document
\end{description}

\makeappendix
\makebackmatter
\end{document}
